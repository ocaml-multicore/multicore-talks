\documentclass{beamer}
\usepackage{graphicx}
\usepackage{verbatim}
\usepackage{alltt}
\usepackage{xcolor}
\usepackage{listings}
\usepackage{hyperref}
\setbeamertemplate{navigation symbols}{}%remove navigation symbols

\renewcommand\UrlFont{\color{blue}}

\newcommand\inputml[1]{\lstinputlisting[language={[Objective]Caml}]{#1}}

\lstdefinelanguage{OCaml}{
  keywords={
    and,as,assert,asr,begin,class,constraint,do,done,downto,effect,else,end,exception,
    external,false,for,fun,function,functor,if,implicit,in,include,inherit,initializer,
    land,lazy,let,lor,lsl,lsr,lxor,macro,match,method,mod,module,mutable,new,object,
    of,open,or,private,rec,sig,struct,then,to,true,try,type,val,virtual,when,
    with,while},
  comment=[s]{(*\ }{\ *)},
}

\definecolor{darkgreen}{rgb}{0,0.2,0}
\definecolor{darkblue}{rgb}{0.1,0.1,0.8}
\definecolor{darkbrown}{rgb}{0.5,0.3,0.0}
\definecolor{grey}{rgb}{0.5,0.5,0.5}
\definecolor{darkgrey}{rgb}{0.2,0.2,0.2}

\lstdefinestyle{ocaml}{
  basicstyle=\ttfamily, % \small
  basewidth=0.5em,
  commentstyle=\color{darkgreen},
  escapeinside={(**}{)},
  keywordstyle=\color{darkblue},
  language=OCaml,
  morekeywords={macro},
  stringstyle=\color{blue},
  showstringspaces=false,
  mathescape=true,
  moredelim=**[is][]{?}{?},
  moredelim=**[is][]{&}{&},
}

\lstdefinestyle{output}{
  basicstyle=\ttfamily\small,
  basewidth=0.5em,
}

\lstset{literate=%
{->}{{$\to$}}2
{...}{{$\ldots$}}2
}

\newcommand\mlkeyword[1]{{\ttfamily\color{darkblue} #1}}

\title[Effects]{Experiences with Effects}
\author[Thomas Leonard]
{Thomas Leonard\and Craig Ferguson\and Patrick Ferris\and Sadiq Jaffer\and Tom Kelly\and KC Sivaramakrishnan\and Anil Madhavapeddy}
\institute{OCaml Labs}
\date[OCaml 2021]{The OCaml Users and Developers Workshop, Aug 2021}

\begin{document}

\definecolor{grey}{gray}{0.6}

\frame{\titlepage}

% \begin{frame}
% \frametitle{Table of Contents}
% \tableofcontents
% \end{frame}

\begin{frame}
	\frametitle{Overview}
	\begin{itemize}
		\item {\color{grey}Domains / }effects {\color{grey}/ typed effects}
		\item Introduction to effects
		\item Case study: Converting the Angstrom parser
		\item Eio concurrency library
	\end{itemize}
\end{frame}

\begin{frame}[fragile]
	\frametitle{Introduction to effects}
	\begin{columns}[t]
		\begin{column}{4.5cm}
	\begin{itemize}
		\item Resumable exceptions
		\item Multiple stacks
	\end{itemize}
		\end{column}
		\begin{column}{5cm}
\begin{lstlisting}[style=ocaml]
effect Foo : int -> int

try
  println "step 1";
  let x = perform (Foo 2) in
  println "step %d" x
with effect (Foo n) k ->
  println "step %d" n;
  continue k (n + 1)
\end{lstlisting}
		\end{column}
	\end{columns}
\end{frame}

\begin{frame}
	\frametitle{Advantages of effects}
	\begin{itemize}
		\item No difference between sequential and concurrent code.
			\begin{itemize}
				\item No special monad syntax.
				\item Can use \mlkeyword{try}, \mlkeyword{match}, \mlkeyword{while}, etc.
				\item No separate lwt or async versions of code.
			\end{itemize}
		\item No heap allocations needed to simulate a stack.
		\item A real stack means backtraces and profiling tools work.
	\end{itemize}
\end{frame}

\begin{frame}
	\frametitle{Case study: Angstrom}

	\url{https://github.com/inhabitedtype/angstrom/}
	\bigskip
	\begin{itemize}
		\item A library for writing parsers
		\item Designed for network protocols
		\item Strong focus on performance
	\end{itemize}
\end{frame}

\begin{frame}[fragile]
	\frametitle{A toy parser}
\begin{lstlisting}[style=ocaml]
type 'a parser = state -> 'a

let any_char state =
  ensure 1 state;
  let c = Input.unsafe_get_char state.input state.pos in
  state.pos <- state.pos + 1;
  c

let (*>) a b state =
  let _ = a state in
  b state
\end{lstlisting}
\end{frame}

\begin{frame}[fragile]
	\frametitle{The Angstrom parser type}
\begin{lstlisting}[style=ocaml,basicstyle=\ttfamily\small]
module State = struct
  type 'a t =
    | Partial of 'a partial
    | Lazy    of 'a t Lazy.t
    | Done    of int * 'a
    | Fail    of int * string list * string
  and 'a partial =
    { committed : int;
      continue  : Bigstringaf.t ->
        off:int -> len:int -> More.t -> 'a t }
end
type 'a with_state = Input.t -> int -> More.t -> 'a
type 'a failure =
  (string list -> string -> 'a State.t) with_state
type ('a, 'r) success = ('a -> 'r State.t) with_state
type 'a parser = { run : 'r.
  ('r failure -> ('a, 'r) success -> 'r State.t) with_state
}
\end{lstlisting}
\end{frame}

\begin{frame}[fragile]
	\frametitle{Angstrom parsers}
\begin{lstlisting}[style=ocaml,basicstyle=\ttfamily\small]
let any_char =
  ensure 1 { run = fun input pos more _fail succ ->
      succ input (pos + 1) more
        (Input.unsafe_get_char input pos)
  }

let (*>) a b =
  { run = fun input pos more fail succ ->
    let succ' input' pos' more' _ =
      b.run input' pos' more' fail succ in
    a.run input pos more fail succ'
  }
\end{lstlisting}
\end{frame}

\begin{frame}[fragile]
	\frametitle{Angstrom : effects branch}
\url{https://github.com/talex5/angstrom/tree/effects}
\bigskip
\begin{lstlisting}[style=ocaml]
type 'a parser = state -> 'a

let any_char state =
  ensure 1 state;
  let c = Input.unsafe_get_char state.input state.pos in
  state.pos <- state.pos + 1;
  c

let (*>) a b state =
  let _ = a state in
  b state
\end{lstlisting}
\end{frame}

\begin{frame}[fragile]
	\frametitle{Parser micro-benchmark}
\begin{lstlisting}[style=ocaml]
let parser = skip_many any_char
\end{lstlisting}
\bigskip
\begin{table}
\begin{tabular}{l|rrrrr}
          & Time     & MinWrds &  MajWrds \\
\hline
Callbacks  & 750.63ms & 160.04Mw & 8,9944.00kw \\
\uncover<2>{Callbacks'} & \uncover<2>{180.73ms} & \uncover<2>{220.01Mw} &   \uncover<2>{9,659.00w} \\
Effects    &  57.81ms &        - &          -
\end{tabular}
\end{table}
\bigskip
\uncover<1>{1}3 times faster!
\pause
\end{frame}

\begin{frame}[fragile]
	\frametitle{Realistic parser benchmark}
Parsing an HTTP request shows smaller gains:
\bigskip
\begin{table}
\begin{tabular}{l|rrrrr}
          & Time     & MinWrds &  MajWrds \\
\hline
Callbacks  &  60.30ms &  9.28Mw & 102.08kw \\
Effects    &  50.71ms &  2.13Mw &  606.30w
\end{tabular}
\end{table}
\end{frame}

\begin{frame}[fragile]
	\frametitle{Using effects for backwards compatibility}
\begin{lstlisting}[style=ocaml]
effect Read : int -> state
let read c = perform (Read c)

let parse p =
  let buffering = Buffering.create () in
  try Unbuffered.parse ~read p
  with effect (Read committed) k ->
    Buffering.shift buffering committed;
    Partial (fun input ->
      Buffering.feed_input buffering input;
      continue k (Buffering.for_reading buffering)
    )
\end{lstlisting}
(simplified)
\end{frame}

\begin{frame}
	\frametitle{Angstrom summary}
	\begin{itemize}
		\item Slightly faster
		\item Much simpler code
		\item No effects in interface
		\item Can convert between callbacks and effects easily
	\end{itemize}
\end{frame}

\begin{frame}
	\frametitle{Eio : an IO library using effects for concurrency}
	\begin{itemize}
		\item Alternative to Lwt and Async
		\item Generic API that performs effects
		\item Cross-platform libuv effect handler
		\item High-performance io-uring handler for Linux
	\end{itemize}
\end{frame}

\begin{frame}[fragile]
	\frametitle{Eio example}
\begin{lstlisting}[style=ocaml]
let handle_connection =
  Httpaf_eio.Server.create_connection_handler
    ~config
    ~request_handler
    ~error_handler

let main ~net =
  Switch.top @@ fun sw ->
  let socket = Eio.Net.listen ~sw net (`Tcp (host, port))
    ~reuse_addr:true
    ~backlog:1000 
  in
  while true do
    Eio.Net.accept_sub ~sw socket handle_connection
      ~on_error:log_connection_error 
  done
\end{lstlisting}
\end{frame}

\begin{frame}
	\frametitle{HTTP benchmark}
	\includegraphics[width=\textwidth]{rps-graph.png}
	100 concurrent connections. Servers limited to 1 core.
\end{frame}

\begin{frame}
	\frametitle{Eio : other features}
	\begin{itemize}
		\item Structured concurrency
		\item OCaps security model
		\item Tracing support
		\item Supports multiple cores
		\item Still experimental
	\end{itemize}
	\includegraphics[width=\textwidth]{trace.png}
\end{frame}

\begin{frame}
	\frametitle{Summary}
	\begin{itemize}
		\item Concurrency with effects works very well
		\item Effects have very good performance
		\item No bugs found in effects system during testing
	\end{itemize}
	\bigskip
	\url{https://github.com/ocaml-multicore/eio} documentation shows how to try out OCaml effects.
\end{frame}

% Backup slides

\begin{frame}[fragile]
	\frametitle{Lwt example}
\begin{lstlisting}[style=ocaml]
let foo ~stdin total =
  let* n = Lwt_io.read_line stdin in
  Lwt_io.printlf "n/total = %d"
    (int_of_string n / total)
\end{lstlisting}
\begin{lstlisting}[style=output]
Fatal error: exception Division_by_zero
Raised at Lwt_example.foo in file "lwt_example.ml", line 6
Called from Lwt.[...].callback in file "src/core/lwt.ml", ...
\end{lstlisting}
	\begin{itemize}
		\item Backtrace doesn't say what called \verb|foo|
		\item Closure with \verb|total| allocated on the heap
	\end{itemize}
\end{frame}

\begin{frame}[fragile]
	\frametitle{Eio example}
\begin{lstlisting}[style=ocaml]
let foo ~stdin total =
  let n = read_line stdin in
  traceln "n/total = %d"
    (int_of_string n / total)
\end{lstlisting}
\begin{lstlisting}[style=output]
Fatal error: exception Division_by_zero
Raised at Eio_example.foo in file "eio_example.ml", line 11
Called from Eio_example.bar in file "eio_example.ml", line 15
...
\end{lstlisting}
\end{frame}


\end{document}
